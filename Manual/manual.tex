\documentclass[11pt, fullpage, a4paper]{article}
%%%Packages 
\usepackage{hyperref} % voor website link

\usepackage{apacite}  % apacite
\usepackage{amssymb, amsmath, amsthm} % symbols and math
\usepackage{graphicx, latexsym}

\usepackage{listings} % voor R code

\usepackage{multirow} % om meerdere rijen in een tabel te bevatten
\usepackage{subfig} % om meerdere figuren in 1 figuur te doen
\usepackage{bibentry} % om referentie in text te kunnen gebruiken
\usepackage{cancel} % om door te kunnen strepen



\newcommand{\Hmi}{$H_{m}^{i} $ }
\newcommand{\Heeni}{\(H_{1}^{i}\) }
\newcommand{\Htweei}{$H_{2}^{i} $ }
\newcommand{\Hdriei}{$H_{3}^{i} $ }
\newcommand{\Hm}{$H_{m} $ }
\newcommand{\Heen}{\(H_{1}\) }
\newcommand{\Htwee}{$H_{2} $ }
\newcommand{\Hdrie}{$H_{3} $ }
\newcommand{\Hu}{$H_{u}$ }
\newcommand{\Hn}{$H_{m'} $ }
\newcommand{\Hmc}{$H_{\cancel{m}} $ }
\newcommand{\Heenci}{$H_{\cancel{1}}^{i} $ }
\newcommand{\Htweeci}{$H_{-2}^{i} $ }
\newcommand{\Hdrieci}{$H_{-3}^{i} $ }
\newcommand{\Hui}{$H_{u}^{i }$ }
\newcommand{\Hni}{$H_{m'}^{i} $ }
\newcommand{\Hmci}{$H_{\cancel{m}}^{i} $ }
\newcommand{\BFmni}{$BF_{mm'}^{i} $ }
\newcommand{\BFmci}{$BF_{m\cancel{m}}^{i} $ }
\newcommand{\BFmui}{$BF_{mu}^{i} $ }
\newcommand{\gBFmni}{$\text{gP-BF}_{mm'} $ }
\newcommand{\gBFmci}{$\text{gP-BF}_{m\cancel{m}} $ }
\newcommand{\gBFmui}{$\text{gP-BF}_{mu} $ }
\newcommand{\xij}{ $ x_{j}^{i} $ }
\newcommand{\piij}{ $ \pi_{j}^{i} $ }
\newcommand{\zedintext}{\citeauthor{zedelius11} \citeyear{zedelius11}}

\title{Manual OneForAll to analyze individual hypotheses and evaluate whether individuals are homogeneous}

\author{Fayette Klaassen}

\date{\today}



\begin{document}
	
	\maketitle	

%This manual contains three parts. 
%Section~\ref{sec:1} describes how the Shiny Application `OneForAll' can be downloaded, that is developed alongside the paper \textit{All for One or Some for All?  Evaluating Informative Hypotheses for Multiple $N=1$ Studies} \cite{klaassenBSES}.
%The requirements for running this program are listed, and it is explained how the program can be executed.
%The program itself consists of two parts.
%The first part of the program, where a Study design analysis can be executed, is described in Section~\ref{sec:3}.
%Section~\ref{sec:2} contains the manual for the second part of the program, where own data can be analyzed.
This manual describes how the Shiny Application `OneForAll' can be used.
A stable link to the app can be found on \href{http://github.com/fayetteklaassen/OneForAll}{http://github.com/fayetteklaassen/OneForAll}.
The application can be run on any computer with an internet connection.
By using the application, you agree to the Terms of Usage, as displayed on the starting screen of the app.
This application allows you to evaluate informative hypotheses for multiple N=1 studies of your own data.
If you want to execute a simulation study (like presented in the paper), please contact the author at klaassen.fayette@gmail.com for R code or a Shiny application you can run locally on your own computer.
	
%	\section{Download, install and start OneForAll}
%	\label{sec:1}
%	This section only describes how to install the software on a Windows computer. 
%	It has not been tested on other operating systems.
%	\subsection{Download files}
%	All relevant files for the application can be downloaded on
%	
%	\noindent \href{http://github.com/fayetteklaassen/OneForAll}{http://github.com/fayetteklaassen/OneForAll}
%	This repository contains a read-me file.
%	\subsection{Download R}
%	In order to be able to run the code, a version of R should be installed on your computer.
%	This code was developed in R 3.3.1, and it is recommended to use this or a newer version of R to run this code.
%	Using version 3.3.1 ensures that the same version of functions is used.
%	How to install a current version of R can be found on \href{https://cran.r-project.org/bin/windows/base/}{the CRAN website}.
%	Optionally R Studio can be installed, which is a more user-friendly interface of R and can be downloaded at the \href{https://www.rstudio.com/products/rstudio/download3/}{site of R Studio}.
%	\subsection{Download Rtools}
%	In order to run the application and compile the code, the package Rtools should be installed, which can be downloaded \href{https://cran.r-project.org/bin/windows/Rtools/}{here}.
%	\subsection{Install packages}
%	The final step before the Shiny program can be used, is to install the required packages. The repository contains a file \textit{Rcode/installpackages.R} that, when run, installs all required packages.
%	
%	\textit{R} Within the program R, you can open the file \textit{installpackages.R}, select all lines of code and press 'Run line or selection', the third button on the top.
%	\textit{R Studio} In R Studio, you can open the file \textit{installpackages.R} and press Ctrl+Shift+S, or select all lines of code and press Ctrl+Enter.
%	\subsection{Run the App}
%	To run the application, you should first open R or R Studio and you run the code \begin{lstlisting}[language=R]
%		library("shiny")
%		\end{lstlisting}
%		\begin{lstlisting}[language=R]
%		runApp("<directory>/OneForAll")	\end{lstlisting}
%		where 
%\textit{ $<$directory$>$} should be replaced by the directory of the folder OneForAll.
%Running this code will open the application in a new screen.
%The first time you open the program, you should wait about a minute for the code to compile.
%When loaded, the screen should look like Figure~\ref{fig:startscreen}.
%
%	The application has two separate parts that can be accessed through the top menu: \textit{Simulate and plot}, which is discussed in Section~\ref{sec:3} and \textit{Analyze own data}, which will be discussed in Section~\ref{sec:2}.
%	
%	\newpage
%		
%		\section{Study design analysis}
%		\label{sec:3}
%		This section provides detailed explanation on how to execute a study design analysis.
%		In the application, two tabs are visible: the Input section, where values for all variable components can be chosen, and the Plot section, where the results can be inspected visually.
%		The Input consists of three steps.
%		\subsection{Input - Step 1: Number of conditions and hypotheses}
%		\label{subsec:constraints}
%		Here you can specify the number of conditions you are considering from the drop down menu.
%		Furthermore, you can choose how you want to specify your hypotheses. 
%		Three options are available.
%		Below examples are provided on how to use this option, using the hypotheses specified in Table~\ref{tab:hypotheses}.
%			\begin{table}[b]
%				\caption{Possible specifications for 6 hypotheses.}
%				\label{tab:hypotheses}
%				\begin{tabular}{l l l l l }
%					Hypothesis & & Using $>$ & Using $R$ & Default \\
%					$H_{1}^{i}: \pi_{1}^{i} >  \pi_{2}^{i} >  \pi_{3}^{i} >  \pi_{4}^{i}>  \pi_{5}^{i} >  \pi_{6}^{i}$ && $\checkmark$ & $\checkmark$ & $\checkmark$ \\
%					
%					$H_{2}^{i}: \pi_{1}^{i} + \pi_{2}^{i} >  \pi_{3}^{i} + \pi_{4}^{i}>  \pi_{5}^{i} + \pi_{6}^{i}$ && $\checkmark$ &$\checkmark$ &$\checkmark$\\
%					
%					$H_{3}^{i}: \pi_{1}^{i} + \pi_{2}^{i} + \pi_{3}^{i} > \pi_{4}^{i} + \pi_{5}^{i} + \pi_{6}^{i}$  && x &$\checkmark$ &x\\
%					
%					$H_{4}^{i}:  \pi_{1}^{i} >  \pi_{2}^{i} >  \pi_{3}^{i} >  \pi_{4}^{i}>  \pi_{6}^{i} >  \pi_{5}^{i}$  && $\checkmark$ &$\checkmark$ &$\checkmark$\\
%					
%					$H_{5}^{i}:  \pi_{1}^{i} >  \pi_{3}^{i} >  \pi_{2}^{i} >  \pi_{4}^{i}>  \pi_{6}^{i} >  \pi_{5}^{i}$  && $\checkmark$ &$\checkmark$ &x\\	
%					
%				\end{tabular}
%			\end{table}
%
%	\begin{itemize}
%		\item \textit{Option 1: Using $>$. } This option requires that for each hypothesis you want to consider, you specify each constraint using $>$ and separate constraints with a comma.
%		Each hypothesis is specified on its own line.
%		The parameters of interest are the success probabilities in the experimental conditions.
%		They can be referred to by a number that corresponds to the column number of that condition in the data.
%		Two types of constraints can be specified: a constraint between two parameters (e.g.``1$>$2"), or a constraint between two combinations of two parameters, separated by a $'+'$ (e.g.``1+2$>$3+4"). 
%		Note that one parameter cannot be on both sides of the constraint (e.g. "1+2 $>$ 2+3" is not allowed).
%		Figure~\ref{fig:ineq} specifies all hypotheses from Table~\ref{tab:hypotheses} that can be specified using this option.
%		\item \textit{Option 2: Using constraint matrix.} This options allows the user to specify a constraint matrix for each hypothesis.
%		For more details on a constraint matrix, see \citeauthor{mulder12} \citeyear{mulder12} for example. 
%		The first line should specify how many hypotheses $M$ are specified, and each hypothesis $m = 1,...,M$ should start with a line specifying the number of constraints (rows) in $R_{m}$.
%		Each constraint matrix contains $J+1$ columns, where $J$ is the number of conditions. 
%		The first $J$ columns specify the constraint matrix, and the last additional column should contain the contrast vector $r$.
%		With this option more complex hypotheses can be specified.
%		Figure~\ref{fig:R} shows how $H_{1}^{i}$ and $H_{3}^{i}$ could be specified using $R$, Option 2. 
%		Option 2 is more flexible than Option 1, but as can be seen in Figures~\ref{fig:ineq} and \ref{fig:R}, Option 1 is more straightforward to specify, if the hypotheses allow for this option.
%		\item \textit{Option 3: Default.} This option is only available for an even number of conditions, and specifies automatically three hypotheses: $H_{1}^{i}: \pi_{1}^{i} > \pi_{2}^{i} > ... > \pi_{J}^{i}$, a full ordered hypothesis, where $J\geq4$, $H_{2}^{i}: \pi_{1}^{i} > \pi_{2}^{i} > ... > \pi_{J}^{i}> \pi_{J-1}^{i}$, that only deviates from $H_{1}^{i}$ because the last two parameters are reversed in the ordering, and finally $H_{3}^{i}:\pi_{1}^{i}+ \pi_{2}^{i} > ... > \pi_{J-1}^{i} +\pi_{J}^{i}$, a full ordered hypothesis of each adjoining pair of parameters.
%	\end{itemize}
%	
%
%		
%	After all hypotheses of interest are specified, the hypotheses can be submitted and checked by pressing the button \textit{Submit constraints for checking}.
%	This will result in a text message below the button specifying the number of hypotheses and showing the hypotheses.
%	The program is only able to reprint relatively simple hypotheses in full form (Option 1 and Option 3), but is always accurate in referring the number of formulated hypotheses, and is therefore valuable to consult even if Option 2 is applied.
%	The user is asked to check the hypothesis themselves if more complicated hypotheses are formulated under Option 2.
%	If any syntactic mistakes are made (wrong number of columns, invalid characters) an error message will appear.
%	
%	\subsection{Optional choices}
%	After submitting the hypotheses of interest, some optional choices can be selected: the Bayes factors to be computed, the P-populations to simulate, and the proportion for the mixture populations.
%	
%	\textit{Step 1a: Bayes factors}
%	Here all possible Bayes factors are presented.
%	A $0$ after a number specifies the Bayes factor against the unconstrained hypothesis - \textit{not} against the null hypothesis.
%	Twice the same number indicates the Bayes factor of that hypothesis against its complement.
%	By default, only Bayes factors for the specified hypotheses against the unconstrained hypothesis are selected.
%	Note that selecting more Bayes factors increases computational time, so only choose those Bayes factors of interest.
%	
%	\textit{Step 1b: P-populations}
%	Here all possible P-populations are presented.
%	Pop0 denotes the unconstrained P-population, and is by default not selected.
%	Pop$m$c denotes the complement of hypothesis $m$.
%	
%	\textit{Step 1c: Mixture proportion}
%	Here it is possible to choose whether a mixture population should be created.
%	If the box is checked, for all models $m$ where both Pop$m$ and Pop$m$c are selected as populations to sample from, a mixture population is formed.
%	By default the proportion in this mixture population is $.5$ of the population adhering to $H_m$ and the remaining $.5$ to $H_{\cancel{m}}$.
%	
%	\subsection{Step 2: Choose R and P}
%	Here the values to consider for the number of replications $R$ and the sample size $P$.
%	The values should be separated by a space.
%	
%	\subsection{Step 3: Simulation details}
%	Here some simulation details should be selected.
%	First, the number of posterior samples to compute the Bayes factor should be specified.
%	By default, this value is $1,000$.
%	For more accuracy, this can be increased to $10,000$, at the cost of computational speed.
%	
%	Second, the size of the simulated P-populations should be specified.
%	This is by default $1,000$, and can be increased to $10,000$, at the cost of computational speed.
%	
%	Third, the number of samples of size $P$ taken from the P-population should be specified.
%	This is by default $1,000$, and can be decreased for increased speed.
%		
%		Finally, a title for this simulation should be specified.
%		A new folder with this title will be created within the application directory, where the simulation is saved.
%		
%		By pressing the button 'Go', the simulation is started. 
%		A progress bar gives a rough indication of the part of the simulation already performed.
%		Note that the computation might take a while: as many populations should be created as the product of number of replications considered and P-populations considered.
%		It is advised to think carefully about which populations to consider in the sensitivity analysis, and which Bayes factor(s) are of interest.
%		Increasing the number of choices in Bayes factors and in the number of populations can add up in computational time.
%		
%		Af
%		
%		\subsection{Plot}
%		The tab \textit{Plot} can be used to view the results of executed simulations (current and past) and the simulations used for the paper.
%		In this panel, you can select from all simulation folders created in the directory of the application.
%		When a folder is selected, you can choose which Bayes factor to plot, whether the $95\%$ intervals should be depicted, and whether the ER and the SR should be depicted.
%		The folder \textit{Datasets} can be selected to view the results shown in the paper.
%		Finally, the plot can be downloaded using the download button.
%	
%	\newpage
	\section{Analyze own data}
	\label{sec:2}
	This section describes each of the steps required to analyze own data in the tab \textit{Analyze own data} within the Shiny Application OneForAll. This item consists of three options from the menu: \textit{Settings and load data}, \textit{Individual Bayes factors}, and \textit{GPBF output}. The first will be discussed in detail, while the other two can be used to view the results.
	\subsection{Step 1: Data and hypotheses}
	Step 1 is to select the data file to be used for analysis.
	You can choose to use the example data from \citeauthor{zedelius11} \citeyear{zedelius11} (as described in the paper \href{https://doi.org/10.3758/s13428-017-0992-5}{One for all or some for all? Evaluting informative hypotheses for multiple N=1 studies} \cite{klaassenBSES}).
	Alternatively, you can upload your own data file to be analyzed.
	This file should be a \textit{.txt} with as many rows as persons or cases, and per row the entries for each condition, separated by a space or tab (white space).
	Each entry in the file should be an integer, describing the number of successes in each condition.
	The rows and columns should not be numbered or labeled. 
	SPSS data can be saved as a Tab delimited .dat file (without row and column names), and the .dat extension must be manually changed to .txt .
	When the file is selected, a preview of the data is visible, together with a description of the number of conditions and the number of participants. 
	If these numbers are correct, you can continue.
	If not, the data file was not in the right format. 
	Common problems are that the first row contains column names (you can just delete this row), or strange symbols in the first entry, which can also be deleted.
	Next, the number of replications used in the experiment should be given.

	\subsection{Step 2: Number of conditions and hypotheses}
\label{subsec:constraints}	
	Step 2 is to define the constraints of the hypotheses considered.
	Three options are available.
		Below examples are provided on how to use this option, using the hypotheses specified in Table~\ref{tab:hypotheses}.
			\begin{table}[b]
				\caption{Possible specifications for 6 hypotheses.}
				\label{tab:hypotheses}
				\begin{tabular}{l l l l l }
					Hypothesis & & Using $>$ & Using $R$ & Default \\
					$H_{1}^{i}: \pi_{1}^{i} >  \pi_{2}^{i} >  \pi_{3}^{i} >  \pi_{4}^{i}>  \pi_{5}^{i} >  \pi_{6}^{i}$ && $\checkmark$ & $\checkmark$ & $\checkmark$ \\
					
					$H_{2}^{i}: \pi_{1}^{i} + \pi_{2}^{i} >  \pi_{3}^{i} + \pi_{4}^{i}>  \pi_{5}^{i} + \pi_{6}^{i}$ && $\checkmark$ &$\checkmark$ &$\checkmark$\\
					
					$H_{3}^{i}: \pi_{1}^{i} + \pi_{2}^{i} + \pi_{3}^{i} > \pi_{4}^{i} + \pi_{5}^{i} + \pi_{6}^{i}$  && x &$\checkmark$ &x\\
					
					$H_{4}^{i}:  \pi_{1}^{i} >  \pi_{2}^{i} >  \pi_{3}^{i} >  \pi_{4}^{i}>  \pi_{6}^{i} >  \pi_{5}^{i}$  && $\checkmark$ &$\checkmark$ &$\checkmark$\\
					
					$H_{5}^{i}:  \pi_{1}^{i} >  \pi_{3}^{i} >  \pi_{2}^{i} >  \pi_{4}^{i}>  \pi_{6}^{i} >  \pi_{5}^{i}$  && $\checkmark$ &$\checkmark$ &x\\	
					
				\end{tabular}
			\end{table}

	\begin{itemize}
		\item \textit{Option 1: Using $>$. } This option requires that for each hypothesis you want to consider, you specify each constraint using $>$ and separate constraints with a comma.
		Each hypothesis is specified on its own line.
		The parameters of interest are the success probabilities in the experimental conditions.
		They can be referred to by a number that corresponds to the column number of that condition in the data.
		Two types of constraints can be specified: a constraint between two parameters (e.g.``1$>$2"), or a constraint between two combinations of two parameters, separated by a $'+'$ (e.g.``1+2$>$3+4"). 
		Note that one parameter cannot be on both sides of the constraint (e.g. "1+2 $>$ 2+3" is not allowed).
		Figure~\ref{fig:ineq} specifies all hypotheses from Table~\ref{tab:hypotheses} that can be specified using this option.
		\item \textit{Option 2: Using constraint matrix.} This options allows the user to specify a constraint matrix for each hypothesis.
		For more details on a constraint matrix, see \citeauthor{mulder12} \citeyear{mulder12} for example. 
		The first line should specify how many hypotheses $M$ are specified, and each hypothesis $m = 1,...,M$ should start with a line specifying the number of constraints (rows) in $R_{m}$.
		Each constraint matrix contains $J+1$ columns, where $J$ is the number of conditions. 
		The first $J$ columns specify the constraint matrix, and the last additional column should contain the contrast vector $r$.
		With this option more complex hypotheses can be specified.
		Figure~\ref{fig:R} shows how $H_{1}^{i}$ and $H_{3}^{i}$ could be specified using $R$, Option 2. 
		Option 2 is more flexible than Option 1, but as can be seen in Figures~\ref{fig:ineq} and \ref{fig:R}, Option 1 is more straightforward to specify, if the hypotheses allow for this option.
		\item \textit{Option 3: Default.} This option is only available for an even number of conditions, and specifies automatically three hypotheses: $H_{1}^{i}: \pi_{1}^{i} > \pi_{2}^{i} > ... > \pi_{J}^{i}$, a full ordered hypothesis, where $J\geq4$, $H_{2}^{i}: \pi_{1}^{i} > \pi_{2}^{i} > ... > \pi_{J}^{i}> \pi_{J-1}^{i}$, that only deviates from $H_{1}^{i}$ because the last two parameters are reversed in the ordering, and finally $H_{3}^{i}:\pi_{1}^{i}+ \pi_{2}^{i} > ... > \pi_{J-1}^{i} +\pi_{J}^{i}$, a full ordered hypothesis of each adjoining pair of parameters.
	\end{itemize}
	
	When the constraints are submitted, the third step is to specify which Bayes factors should be computed.
	The options available are all combinations of the hypotheses specified, and each hypothesis against its complement and the unconstrained hypothesis.
	By pressing the button `Check constraints' the constraints are checked, and a textbox is returned with the hypothesis as processed by the app. 
	If something is incorrect here, please re-enter your constraints.
	
	\subsection{Step 3: Computation details}
The number of iterations required for the computation of the Bayes factor. 
	By default, this value is $10,000$. 
	Decreasing this value will increase the speed of computation, but particularly for larger number of conditions (say $8$), decrease the precision of the Bayes factor computation.
	You can enter and adjust the computation seed, for reproducibility of your results.
	
	By then pressing the button `Execute analysis' (appears if the constraints are filled in and checked), the computation will start. 
	A pop-up will appear in the bottom right corner to indicate that the computation is busy, and a notification `Analysis finished' will appear under the button when ready.
	Then, you can access the other two tabs to view the results
%\begin{figure}[h]
%	\caption{Start screen application}
%	\label{fig:startscreen}
%	\includegraphics[width = \linewidth]{pictures/startScreen.jpg}
%\end{figure}
				\begin{figure}
					\caption{Option 1: Using $>$}
					\label{fig:ineq}
					\includegraphics[]{pictures/option1.jpg}										
				\end{figure}
				
				\begin{figure}
					\caption{Option 2: Using $R$}
					\label{fig:R}
					\includegraphics[]{pictures/option2.jpg}
				\end{figure}

	
	
	
	\clearpage
	\newpage
	\bibliographystyle{apacite}
	\bibliography{D:/Surfdrive/references}		
\end{document}